%#!platex -kanji=%k
\documentclass[uplatex]{jsarticle}
\title{レポートなりよ}
\author{c103333\\公益太郎}
\begin{document}
\maketitle
\section{はじめに}

これがレポートのソースなり。なんやよく分からんがとりあえず文章をたっぷり
書いてけばいいらしい。

 \section{色々な記号}
 
 記号もいろいろ出るらしい。

  \subsection{トランプの記号}

  \begin{itemize}
   \item $\heartsuit$ は \$$\backslash$heartsuit\$ と書く
   \item $\spadesuit$ は \$$\backslash$spadesuit\$ と書く
   \item $\diamondsuit$ は \$$\backslash$diamondsuit\$ と書く
   \item $\clubsuit$ は \$$\backslash$clubesuit\$ と書く
  \end{itemize}

  \subsection{数学の記号}

  \begin{itemize}
   \item $\sum_{k=0}^{\infty}a_{k}$ とかも書けるみたい。まいっか。
  \end{itemize}
  同じようにソースを書いても、上のように一行で書くときと、
  \begin{equation}
   \sum_{k=0}^{\infty}a_{k}
  \end{equation}
  のように行立てのときの見え方が変わるらしい。ま、いっか。

 \section{箇条書き}

 箇条書きは
 \begin{enumerate}
  \item 箇条書きの範囲を\verb|\begin{itemize}〜\end{itemize}| で括って
  \item 各項目を \verb|\item| で書き始める
 \end{enumerate}
 らしい。これって、HTMLの「全体を \verb|<ul>〜</ul>| で括って 
 \verb|<li>| で項目を書く」、のに似てるぞ。というか、実は\LaTeX のほうが
 手本らしいぞ。番号つきの箇条書きは、\verb|itemize| のかわりに
 \verb|enumerate| にする。

\end{document}
